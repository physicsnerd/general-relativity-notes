How can we shape society through revolution? It is an important question.

Einstein published the theory of special relativity in 1905 and the theory of general relativity in 1915. 
That marked an incredible revolution in both physics and modern thought. 
While most don’t even know how general relativity works on an intuitive level, let alone a mathematical level, 
the theory has transformed philosophy, showing that there must have been some beginning to the universe 
(the current theory of which is the Big Bang). 
People have been confronted with a result that contradicts their religion (or lack of one) or confirms it. 
Further, general relativity resulted in the discovery of black holes and more recently gravitational waves. 
These discoveries put both physicists and non-physicists alike in awe of the universe and the mysteries it holds. 
Finally, it has resulted in the biggest conflict in science perhaps ever: the conflict between general relativity and quantum mechanics. 
Scientists still are struggling (though they have made significant process) to figure out a correct theory 
that explains the universe as we know it. So, how can science transform revolution? How can science shape society?
Well, it can (and does) change our thought about our past, our future, and our present. 
It can (and does) put us in awe of our universe. It can (and does) make us wonder how it all began. 
And it can, and does, introduce new technologies into our lives.
