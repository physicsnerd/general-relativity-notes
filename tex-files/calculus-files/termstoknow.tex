\begin{itemize}
\item Slope: slope is defined as rise over run, or, more simply, how "steep" a line is. 
If a line is going "down", it has negative slope; if it is going "up" it has positive slope.
\item Speed: the change in position over time. 
You might be going 70 miles per hour in your car. 
Well, your  position is changing by 70 miles, every hour.
\item Velocity: speed, but with direction. 
For example, you might be biking at a speed of 10 miles per hour, and heading north. 
That's a velocity.
\item Acceleration: the change in velocity or speed over time. 
You might accelerate by a mile per minute, starting from 70 miles per hour. 
After 30 minutes, you'll be going 100 miles per hour, and your
speed will still be increasing. At that point, it's a good idea to
decelerate, or have a negative change in velocity/speed, and hope a
police officer hasn't seen you.
\item Function: something that takes in a number and spits out another
number. We usually write a function as $f(x)$, or $x(m)$, or $g(x)$, or
h(x)...something along those lines. We then say that as "$f$ of $x$". It
just means what a function $f$ evaluates too when you plug in a number $x$.
\item 9.81 m/s: the terminal acceleration of an object. Basically,
let's say you drop an object. It can't accelerate any faster than that
speed, no matter what. Blame gravity.
\item Work: a physics concept (we're not talking about a job here).
Basically, let's say you pick up a book and bring it to the kitchen
table. You just exerted force on an object. You just did work. Note
that in physics, if you haven't moved anything, you haven't done any
work. Basically, the amount of work you've done depends on the force
you've exerted (which in turn depends on mass and acceleration) and
displacement (or how far you've moved an object from it's original
location).
\end{itemize}
