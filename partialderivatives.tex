Here we will examine only one part of multivariable calculus, partial derivatives.
Partial derivatives are actually not too difficult once you know how to do normal derivatives.
The basic principle is this: given a function, take the derivative of it with respect to one variable, while treating the other variables as constants.
How does this work? Well, let's take as an example the function $f(x,y)=3x-2y^4$.
We would first take the derivative of this function with respect to $x$, treating $y$ as a constant. 
So, the derivative of $f_x$ would equal $3$ - the second term vanishes because $y$ is a constant. 
Then, we would take the derivative of the function with respect to $y$, treating $x$ as a constant.
The derivative of $f_y$ would equal, therefore, $-8y^3$. 
So the partial derivative of $f(x,y)$ is $f_x=3, f_y=-8y^3$.
Let's do this again with a more complicated function. Given $f(x,y,z)=xy^2z^3+3yz$, we would get $f_x=y^2z^3$, $f_y=xz^32y+3z$, and $f_z=xy^23z^2+3y$.
The other important thing to mention is how partial derivatives are written. Instead of writing $\frac{d}{dx}$ or $\frac{dx}{dy}$ to represent derivatives (the second representing the derivative of the function $x$ with respect to $y$ and the first representing the derivative of what follows with respect to $x$) we write $\frac{\partial}{\partial x}$ or $\frac{\partial x}{\partial y}$ - the $\partial$ symbol represents the derivative.